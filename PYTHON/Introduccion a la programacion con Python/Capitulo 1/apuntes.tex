\documentclass{article}
\usepackage[utf8]{inputenc}
\usepackage{geometry}
\geometry{
	a4paper,
	total={170mm,257mm},
	left=20mm,
	top=20mm,
}
\usepackage{graphicx}
\usepackage{titling}

\title{Apuntes del libro
}
\author{María Sofía Ojeda Mejía}
\date{Diciembre 2024}

\usepackage{fancyhdr}
\fancypagestyle{plain}{%  the preset of fancyhdr 
	\fancyhf{} % clear all header and footer fields
	\fancyfoot[L]{\thedate}
	\fancyhead[R]{\theauthor}
}
\makeatletter
\def\@maketitle{%
	\newpage
	\null
	\vskip 1em%
	\begin{center}%
		\let \footnote \thanks
		{\LARGE \@title \par}%
		\vskip 1em%
		%{\large \@date}%
	\end{center}%
	\par
	\vskip 1em}
\makeatother

\usepackage{lipsum}
\usepackage{cmbright}

\begin{document}
	
	\maketitle
	
	\noindent\begin{tabular}{@{}ll}
  & \theauthor\\

	\end{tabular}
	
	\section*{Lección 1:}
	Python es un lenguaje de programación, lo que significa que
	
	\section*{Lección 2:}
lkefafkjdkhfkjahefkhajf
	
	\section*{Lección 3:}
kjshckhfkdhsdkjhfksjhfkjjhsdkfhdskjjfhsdkjhfkshf
	
	
	
\end{document}

